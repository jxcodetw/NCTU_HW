\documentclass[A4]{article}

\usepackage[english]{babel}
\usepackage[utf8]{inputenc}
\usepackage{amsmath}
\usepackage{graphicx}
\usepackage{float}
\usepackage{enumitem}
\usepackage[toc,page]{appendix}
\usepackage{url}
\usepackage{booktabs}

\title{Evolutionary Computation Homework \#2 Report}

\author{Chien-Hsun, Lai}

\date{\today}
% ---- 1. Section title -> Section 1: Section title
\makeatletter
% we use \prefix@<level> only if it is defined
\renewcommand{\@seccntformat}[1]{%
  \ifcsname prefix@#1\endcsname
    \csname prefix@#1\endcsname
  \else
    \csname the#1\endcsname\quad
  \fi}
% define \prefix@section
\newcommand\prefix@section{Problem \thesection:}
\makeatother
% -----
\begin{document}
\maketitle

\section{}
\label{sec:p1}

GA uses fitness as the probability of selecting the parent.
Not only the elites are selected.
It maintains a diversity of solution candidates.
ES threats fitness as rank.
Only top-k individuals are selected.
They tends to walk to the best possible solution space.
If the mutation rate (step size) is too small,
it will lose the chance to explorer other area of the solution space.
But if the fitness surface is smooth, ES will run as fast as it can towards a optimal.

\section{}
\label{sec:p2}

They probably won't change.

\section{}
\label{sec:p3}

\begin{table}[H]
\begin{minipage}{.5\linewidth}
	\centering
	\begin{tabular}{@{}llll@{}}
	\toprule
	(1+1)-ES & 0.01 & 0.1    & 1  \\ \midrule
	Run \#1  & 759  & 717781 & -1 \\
	Run \#2  & 796  & 678544 & -1 \\
	Run \#3  & 822  & 122065 & -1 \\
	Run \#4  & 913  & 233446 & -1 \\
	Run \#5  & 819  & 363942 & -1 \\
	Run \#6  & 817  & 21134  & -1 \\
	Run \#7  & 882  & 108376 & -1 \\
	Run \#8  & 884  & 36912  & -1 \\
	Run \#9  & 827  & 9580   & -1 \\
	Run \#10 & 811  & 172545 & -1 \\ \bottomrule
	\end{tabular}
\end{minipage}%
\begin{minipage}{.5\linewidth}
	\begin{tabular}{@{}llll@{}}
	\toprule
	(1,1)-ES & 0.01 & 0.1 & 1  \\ \midrule
	Run \#1  & -1   & -1  & -1 \\
	Run \#2  & -1   & -1  & -1 \\
	Run \#3  & -1   & -1  & -1 \\
	Run \#4  & -1   & -1  & -1 \\
	Run \#5  & -1   & -1  & -1 \\
	Run \#6  & -1   & -1  & -1 \\
	Run \#7  & -1   & -1  & -1 \\
	Run \#8  & -1   & -1  & -1 \\
	Run \#9  & -1   & -1  & -1 \\
	Run \#10 & -1   & -1  & -1 \\ \bottomrule
	\end{tabular}
\end{minipage}
\caption{Fixed step-size for Gaussian mutation}
\end{table}

\section{}
\label{sec:p4}

(1,1)-ES generate 1 child and then use it as the next parent.
It is equivalent to random search. Wherever it mutate, it goes there.
So it is really hard to get to the answer we want (need a lot of luck).
\\
\\
(1+1)-ES select among child and parent.
It only take a move when there are better solution.
If there aren't any better solutions, it'll stay where it is.
But if the mutation step is too big, it'll be hard to randomly get a better child.

\section{}
\label{sec:p5}

Parameters:

$ \tau = exp(-0.00007 * n) $

$ \tau\ensuremath{'} = \frac{1}{\sqrt{2\sqrt{n}}} $

$ \epsilon_0 = 0.001 $

\begin{table}[H]
	\begin{minipage}{.5\linewidth}
		\centering
		\begin{tabular}{@{}llll@{}}
		\toprule
		(1+1)-ES & 0.01 & 0.1  & 1    \\ \midrule
		Run \#1  & 2397 & 3220 & -1   \\
		Run \#2  & 1251 & 859  & 1001 \\
		Run \#3  & 2319 & 2207 & 5197 \\
		Run \#4  & 1000 & 2433 & 1922 \\
		Run \#5  & 5647 & 1999 & 1374 \\
		Run \#6  & 2029 & 2950 & 1377 \\
		Run \#7  & 609  & 939  & -1   \\
		Run \#8  & 1735 & 721  & 1006 \\
		Run \#9  & 1742 & 1080 & -1   \\
		Run \#10 & 3262 & 1389 & 7910 \\ \bottomrule
		\end{tabular}
	\end{minipage}%
	\begin{minipage}{.5\linewidth}
		\begin{tabular}{@{}llll@{}}
		\toprule
		(1,1)-ES & 0.01 & 0.1 & 1  \\ \midrule
		Run \#1  & -1   & -1  & -1 \\
		Run \#2  & -1   & -1  & -1 \\
		Run \#3  & -1   & -1  & -1 \\
		Run \#4  & -1   & -1  & -1 \\
		Run \#5  & -1   & -1  & -1 \\
		Run \#6  & -1   & -1  & -1 \\
		Run \#7  & -1   & -1  & -1 \\
		Run \#8  & -1   & -1  & -1 \\
		Run \#9  & -1   & -1  & -1 \\
		Run \#10 & -1   & -1  & -1 \\ \bottomrule
		\end{tabular}
	\end{minipage}
	\caption{Uncorrelated Gaussian mutation}
\end{table}


\section{}
\label{sec:p6}

Self-adaptation improves the result of (1+1)-ES.
It might due to the decay learning rate. Smaller step search for finer solution.
Sometimes for initial step size = 1, it won't converge.
It might due to the inherited step size.
Although the child get a better score to replace the parent,
the step size might mutate to a large one.
So it is takes a lot of time for learning rate to decay to get another better child.
\\
\\
(1,1)-ES still acts as if it was completely random.

\section{}
\label{sec:p7}

Parameters:

$ G = 30 $

$ a = 0.85 $

\begin{table}[H]
	\begin{minipage}{.5\linewidth}
		\centering
		\begin{tabular}{@{}llll@{}}
		\toprule
		(1+1)-ES & 0.01 & 0.1 & 1   \\ \midrule
		Run \#1  & 506  & 376 & 696 \\
		Run \#2  & 493  & 433 & 617 \\
		Run \#3  & 521  & 358 & 708 \\
		Run \#4  & 539  & 390 & 663 \\
		Run \#5  & 540  & 489 & 695 \\
		Run \#6  & 619  & 334 & 636 \\
		Run \#7  & 454  & 464 & 692 \\
		Run \#8  & 427  & 437 & 690 \\
		Run \#9  & 540  & 351 & 679 \\
		Run \#10 & 464  & 375 & 699 \\ \bottomrule
		\end{tabular}
	\end{minipage}%
	\begin{minipage}{.5\linewidth}
		\begin{tabular}{@{}llll@{}}
		\toprule
		(1,1)-ES & 0.01 & 0.1 & 1  \\ \midrule
		Run \#1  & -1   & -1  & -1 \\
		Run \#2  & -1   & -1  & -1 \\
		Run \#3  & -1   & -1  & -1 \\
		Run \#4  & -1   & -1  & -1 \\
		Run \#5  & -1   & -1  & -1 \\
		Run \#6  & -1   & -1  & -1 \\
		Run \#7  & -1   & -1  & -1 \\
		Run \#8  & -1   & -1  & -1 \\
		Run \#9  & -1   & -1  & -1 \\
		Run \#10 & -1   & -1  & -1 \\ \bottomrule
		\end{tabular}
	\end{minipage}
	\caption{1/5-rule}
\end{table}



\section{}
\label{sec:p8}

1/5-rule with (1+1)-ES out-perform the previous methods.
Clearly, being able to dynamically change the step size helps a lot.
Moreover, 1/5-rule didn't just decay it's step-size.
It'll increase it's step-size to expand the search space or shrink the search space
depending on how current mutation performs.

\begin{appendices}
\chapter{Source code for experiments}\\
\url{https://github.com/jxcodetw/NCTU_GA/tree/master/hw2}
\end{appendices}

\end{document}