\documentclass[A4]{article}

\usepackage[english]{babel}
\usepackage[utf8]{inputenc}
\usepackage{amsmath}
\usepackage{graphicx}
\usepackage{float}
\usepackage{enumitem}
\usepackage[toc,page]{appendix}
\usepackage{url}

\title{Evolutionary Computation Homework \#1 Report}

\author{Chien-Hsun, Lai}

\date{\today}
% ---- 1. Section title -> Section 1: Section title
\makeatletter
% we use \prefix@<level> only if it is defined
\renewcommand{\@seccntformat}[1]{%
  \ifcsname prefix@#1\endcsname
    \csname prefix@#1\endcsname
  \else
    \csname the#1\endcsname\quad
  \fi}
% define \prefix@section
\newcommand\prefix@section{Problem \thesection:}
\makeatother
% -----
\begin{document}
\maketitle

\section{}
\label{sec:p1}

\begin{enumerate}[label=(\alph*)]
\item From a $ 8*8 $ board choose $ 8 $ locations: $ C^{64}_{8} = 4,426,165,368 $.
\item We can use 8 tuples $ \{(x_1, y_1), ... , (x_8, y_8)\} $ to encode the location of each queen.
\item $ 1 \leq x_i, y_i \leq 8 $ There are $ (8*8)^8 = 281,474,976,710,656 $ possible states.
\item Each location of the queen is mapped to a tuple $(x_i, y_i)$.
\end{enumerate}

\section{}
\label{sec:p2}

There are $ 1 / 0.001 = 1000 $ intervals in $ [0, 1] $.
We'll need at least 10 bit (with $ 2^{10} = 1024 $ intervals) to achieve that precision.

\section{}
\label{sec:p3}

\begin{figure}[H]
\centering
\includegraphics[width=0.8\textwidth]{logs/p3_fig.png}
\caption{\label{fig:p3} Result with default parameters.}
\end{figure}

\section{}
\label{sec:p4}

\begin{figure}[H]
\centering
\includegraphics[width=0.8\textwidth]{logs/p4_fig.png}
\caption{\label{fig:p4} Add 1000 to fitness function (fitness is subtracted 1000 in the fig)}
\end{figure}

\section{}
\label{sec:p5}

The original fitnesses range from $[0, 50]$.
After you add 1000 to them, the probability being chosen approaches uniform distribution.
For example, assume there are two individuals with fitness 10 and 50 respectively.
Before adding 1000, the probability the two being chosen are $ 10 / 60 \approx 16.66\% $ and $ 50/60 \approx 83.33\% $.
After adding 1000, it becomes $ 1010/2060 \approx 49\% $ and $ 1050/2060 \approx 51\% $.
So the experiment in Problem 4 won't imporve the fitness over generation.
Those individuals are just mixing there gene randomly.
\\
The figure in Problem 4 shows that the best fitness dropped over generations.
Which indicate that the fitness of all individuals are averaging with each other.
\section{}
\label{sec:p6}

\begin{figure}[H]
\centering
\includegraphics[width=0.8\textwidth]{logs/p6_fig.png}
\caption{\label{fig:p6} Repeat Problem 3 with tournament selection.}
\end{figure}

\section{}
\label{sec:p7}

\begin{figure}[H]
\centering
\includegraphics[width=0.8\textwidth]{logs/p7_fig.png}
\caption{\label{fig:p7} Repeat Problem 4 with tournament selection.}
\end{figure}

\section{}
\label{sec:p8}

Despite the fitness function was different. Both experiments still converged and found the best solution.
After you add $ 1000 $ to the fitness function.
It doesn't affect the comparasion of two individuals.
The rank of each individuals doesn't change.
One with higher fitness still has a higher fitness after adding $ 1000 $.
So experiments in both problem shows very similar results.

\section{}
\label{sec:p9}
From Problem 5 and Problem 8, we can observ that:
\begin{enumerate}
\item Roulette selection depends heavily on the proportionaly difference of each individuals.
\item Tournament selection depends on the rank of each individuals.
\end{enumerate}

\begin{appendices}
\chapter{Source code for experiments}\\
\url{https://github.com/jxcodetw/NCTU_GA/tree/master/hw1}
\end{appendices}

\end{document}